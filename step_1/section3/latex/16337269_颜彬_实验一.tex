%%%%%%%%%%%%%%%%%%%%%%%%%%%%%%%%%%%%%%%%%
% Programming/Coding Assignment
% LaTeX Template
%
% This template has been downloaded from:
% http://www.latextemplates.com
%
% Original author:
% Ted Pavlic (http://www.tedpavlic.com)
%
% Note:
% The \lipsum[#] commands throughout this template generate dummy text
% to fill the template out. These commands should all be removed when 
% writing assignment content.
%
% This template uses a Perl script as an example snippet of code, most other
% languages are also usable. Configure them in the "CODE INCLUSION 
% CONFIGURATION" section.
%
%%%%%%%%%%%%%%%%%%%%%%%%%%%%%%%%%%%%%%%%%

%----------------------------------------------------------------------------------------
%	PACKAGES AND OTHER DOCUMENT CONFIGURATIONS
%----------------------------------------------------------------------------------------

\documentclass[a4paper]{article}

\usepackage{fancyhdr} % Required for custom headers
\usepackage{lastpage} % Required to determine the last page for the footer
\usepackage{extramarks} % Required for headers and footers
\usepackage[usenames,dvipsnames]{color} % Required for custom colors
\usepackage{graphicx} % Required to insert images
\usepackage{listings} % Required for insertion of code
\renewcommand*{\lstlistingname}{代码} % change "Listing <ref> to 代码 <ref>
\usepackage{courier} % Required for the courier font
\usepackage{lipsum} % Used for inserting dummy 'Lorem ipsum' text into the template

\usepackage[UTF8]{ctex} % Required for Chinese character
\usepackage{tocloft} % Required for beautiful toc
\usepackage[hidelinks]{hyperref} % Required for clickable toc
\hypersetup{
    colorlinks,
    citecolor=black,
    filecolor=black,
    linkcolor=black,
    urlcolor=black
}
\usepackage[title]{appendix} % Required for appendix

% Margins
\topmargin=-0.45in
\evensidemargin=0in
\oddsidemargin=0in
\textwidth=6.5in
\textheight=9.0in
\headsep=0.25in

\linespread{1.1} % Line spacing

% Set up the header and footer
\pagestyle{fancy}
\lhead{\hmwkAuthorName} % Top left header
\chead{\hmwkClass\ (\hmwkClassInstructor\ \hmwkClassTime): \hmwkTitle} % Top center head
\rhead{\firstxmark} % Top right header
\lfoot{\lastxmark} % Bottom left footer
\cfoot{} % Bottom center footer
\rfoot{Page\ \thepage\ of\ \protect\pageref{LastPage}} % Bottom right footer
\renewcommand\headrulewidth{0.4pt} % Size of the header rule
\renewcommand\footrulewidth{0.4pt} % Size of the footer rule

\setlength\parindent{0pt} % Removes all indentation from paragraphs

%----------------------------------------------------------------------------------------
%	CODE INCLUSION CONFIGURATION
%----------------------------------------------------------------------------------------

\definecolor{MyDarkGreen}{rgb}{0.0,0.4,0.0} % This is the color used for comments
\lstloadlanguages{Perl} % Load Perl syntax for listings, for a list of other languages supported see: ftp://ftp.tex.ac.uk/tex-archive/macros/latex/contrib/listings/listings.pdf
\lstset{language=Perl, % Use Perl in this example
        frame=single, % Single frame around code
        basicstyle=\small\ttfamily, % Use small true type font
        keywordstyle=[1]\color{Blue}\bf, % Perl functions bold and blue
        keywordstyle=[2]\color{Purple}, % Perl function arguments purple
        keywordstyle=[3]\color{Blue}\underbar, % Custom functions underlined and blue
        identifierstyle=, % Nothing special about identifiers                                         
        commentstyle=\usefont{T1}{pcr}{m}{sl}\color{MyDarkGreen}\small, % Comments small dark green courier font
        stringstyle=\color{Purple}, % Strings are purple
        showstringspaces=false, % Don't put marks in string spaces
        tabsize=5, % 5 spaces per tab
        %
        % Put standard Perl functions not included in the default language here
        morekeywords={rand},
        %
        % Put Perl function parameters here
        morekeywords=[2]{on, off, interp},
        %
        % Put user defined functions here
        morekeywords=[3]{test},
       	%
        morecomment=[l][\color{Blue}]{...}, % Line continuation (...) like blue comment
        numbers=left, % Line numbers on left
        firstnumber=1, % Line numbers start with line 1
        numberstyle=\tiny\color{Blue}, % Line numbers are blue and small
        stepnumber=2, % Line numbers go in steps of 5,
        firstnumber=1
}

% Creates a new command to include a perl script, the first parameter is the filename of the script (without .pl), the second parameter is the caption
\newcommand{\perlscript}[2]{
\begin{itemize}
\item[]\lstinputlisting[caption=#2,label=#1]{#1.pl}
\end{itemize}
}

\newcommand{\shfilescript}[3]{
\begin{itemize}
\item[]\lstinputlisting[caption=#2, label=#1, language=sh]{#3}
\end{itemize}
}
\newcommand{\shscript}[3]{
\begin{itemize}
\item[]\begin{lstlisting}[label=#1, caption=#2] #3 \end{lstlisting}
\end{itemize}
}

%----------------------------------------------------------------------------------------
%	DOCUMENT STRUCTURE COMMANDS
%	Skip this unless you know what you're doing
%----------------------------------------------------------------------------------------

% Header and footer for when a page split occurs within a problem environment
\newcommand{\enterProblemHeader}[1]{
\nobreak\extramarks{#1}{#1 见下页\ldots}\nobreak{} 
\nobreak\extramarks{接上页}{#1 见下页\ldots}\nobreak{}
}

% Header and footer for when a page split occurs between problem environments
\newcommand{\exitProblemHeader}[1]{
\nobreak\extramarks{接上页}{#1 见下页\ldots}\nobreak{}
\nobreak\extramarks{#1}{}\nobreak{}
}
% TODO:code here enable the number before section, but it disable the numbering of problems
%\setcounter{secnumdepth}{0} % Removes default section numbers
\newcounter{homeworkProblemCounter} % Creates a counter to keep track of the number of problems

\newcommand{\homeworkProblemName}{}

\newenvironment{homeworkProblem}[1][Problem \arabic{homeworkProblemCounter}]{ % Makes a new environment called homeworkProblem which takes 1 argument (custom name) but the default is "Problem #"
\stepcounter{homeworkProblemCounter} % Increase counter for number of problems
\renewcommand{\homeworkProblemName}{#1} % Assign \homeworkProblemName the name of the problem
\section{\homeworkProblemName} % Make a section in the document with the custom problem count
\enterProblemHeader{\homeworkProblemName} % Header and footer within the environment
}{
\exitProblemHeader{\homeworkProblemName} % Header and footer after the environment
}

\newcommand{\problemAnswer}[1]{ % Defines the problem answer command with the content as the only argument
\noindent\framebox[\columnwidth][c]{\begin{minipage}{0.98\columnwidth}#1\end{minipage}} % Makes the box around the problem answer and puts the content inside
}

\newcommand{\homeworkSectionName}{}
\newenvironment{homeworkSection}[1]{ % New environment for sections within homework problems, takes 1 argument - the name of the section
\renewcommand{\homeworkSectionName}{#1} % Assign \homeworkSectionName to the name of the section from the environment argument
\subsection{\homeworkSectionName} % Make a subsection with the custom name of the subsection
\enterProblemHeader{\homeworkProblemName\ [\homeworkSectionName]} % Header and footer within the environment
}{
\enterProblemHeader{\homeworkProblemName} % Header and footer after the environment
}

%----------------------------------------------------------------------------------------
%	NAME AND CLASS SECTION
%----------------------------------------------------------------------------------------

\newcommand{\hmwkTitle}{操作系统原理实验\ \#1} % Assignment title
\newcommand{\hmwkDueDate}{Monday,\ March\ 10,\ 2018} % Due date
\newcommand{\hmwkClass}{16级计科\ 7班} % Course/class
\newcommand{\hmwkClassTime}{周一9-10节} % Class/lecture time
\newcommand{\hmwkClassInstructor}{凌应标} % Teacher/lecturer
\newcommand{\hmwkAuthorName}{颜彬} % Your name
\newcommand{\hmwkAuthorId}{16337269} % Your id 

%----------------------------------------------------------------------------------------
%	TITLE PAGE
%----------------------------------------------------------------------------------------

\usepackage{titling}

\title{
\vspace{2in}
\textmd{\textbf{\hmwkClass:\ \hmwkTitle}}\\
\normalsize\vspace{0.1in}\small{Due\ on\ \hmwkDueDate}\\
\vspace{0.1in}\large{\textit{\hmwkClassInstructor\ \hmwkClassTime}}
\vspace{3in}
}

\author{\textbf{\LARGE{\hmwkAuthorName}} \\ \\ \textbf{\LARGE{\hmwkAuthorId}}}
\date{} % Insert date here if you want it to appear below your name
%----------------------------------------------------------------------------------------

\begin{document}
% \begin{titlingpage} % This is for ignore page number in first page. package titling

\maketitle

%----------------------------------------------------------------------------------------
%	TABLE OF CONTENTS
%----------------------------------------------------------------------------------------

\setcounter{tocdepth}{2} % Uncomment this line if you don't want subsections listed in the ToC


\renewcommand{\cftsecleader}{\cftdotfill{\cftdotsep}} % used for dots between <section> and <page>
\renewcommand{\contentsname}{Content} % force the word to be "content
\newpage
\tableofcontents
\addtocontents{toc}{~\hfill\textbf{Page}\par}
\newpage

% below are document body


% To have just one problem per page, simply put a \clearpage after each problem
\section{实验目的}
掌握操作系统的相关概念,理解计算机引导的全部过程。利用工具制作正确引导盘并完成特定任务。\\ 
理解虚拟机的运行方式,了解虚拟机与真实机器的不同之处。熟悉配置、运行虚拟机的基本方法。掌握在虚拟机下运行程序、程序
查错和运行dos镜像\\ 
熟悉相关工具的使用。熟悉32位x86汇编器和二进制文件操作软件的使用。熟悉虚拟镜像的制作方法。
掌握基本的32位x86汇编的编写。掌握汇编程序的编译、运行、调错。掌握硬件级的调试技巧。
\section{实验要求}

\subsection{搭建和应用实验环境}
虚拟机安装,生成一个基本配置的虚拟机 XXXPC 和多个 1.44MB 
容量的虚拟软盘,将其中一个虚拟软盘用 DOS 格式化为 DOS 引导盘,用 WinHex 工具将其中一个虚拟软盘的首扇区填满你的个人信息
\subsection{接管裸机控制权}
设计 ibmpc 的一个引导扇区程序
,程序功能从屏幕左边某行位置45度
角斜向下射出,保持一个可观察的适当速度直线运动,在碰到屏幕边后产生反射,
改变方向运动,如此类推,不断运动。在此基础上,增加你的个性拓展,如同时控制两个运动的轨迹,
或炫酷动态变色,个性画面,如此等等,自由不限。还要在屏幕某个区域以特别的方式显示你的学号姓名
等个人信息。将这个程序的机器吗放到第三张虚拟软盘的首扇区,并用此软盘引导你的pc,直到成功。


\section{实验方案}
\subsection{基础原理}
实验环境是不带有操作系统的裸机。为了做到接管逻辑控制权和运行程序的目的,我们需要写出自己的引导程序(制作一个引导扇区)。
开机后,计算机会首先作自检。若选择从软盘启动,bios会检查软盘的0面0磁道,如果发现它的最后一个字节是0xAA55,则bios会把其视作一个引导扇区,
尝试执行其中的程序。\\ 

引导扇区通常会将操作系统加载进内存中,将控制权完全交给操作系统,由操作系统完成随后的各种调配。在本实验中,引导扇区内包含了
若干代码,用于在屏幕中心高亮显示个人信息,以及完成简单的动画。 \\

即便没有操作系统,本次实验依然可以在终端输出相应的信息,甚至是绘制动画。
0xB8000-0xBFFFF的内存空间是显存地址,共有32KB。向这个地址写入数据可以打印到屏幕上。
对于从0xB800开始的每个字\(16 bits\),其高位将解释为输出内容的样式信息,例如前景色、背景色等;其低位将解释为输出内容的Ascii码。\\

通过反复往显存地址中读写数据,本项目最终完成了附带个人资料和动画的引导程序。



\subsection{实验环境与工具}
\subsubsection{实验环境}
\begin{itemize}
    \item 操作系统 \\ 
    本实验在Linux下完成。采用Ubuntu 16.04
    \item 虚拟机\\
    bochs.它是一款开源且跨平台的 IA-32 模拟器。
\end{itemize}
\subsubsection{相关工具、指令}
\begin{itemize}
    \item 汇编器\\ 
    NASM. NASM是一个轻量级的、模块化的 80x86 和 x86-64 汇编器。它的语法与
    Intel 原语法十分相似,但更加简洁和易读。它对宏有十分强大的支持。
    \item 镜像文件产生工具\\ 
    bximage. 该命令允许生成指定大小的软件镜像。
    \item 二进制写入命令\\ 
    dd. dd 允许指定源文件和目标文件,将源文件的二进制比特写入目标文件中的指定位置。
    \item 二进制文件查看命令\\ 
    xxd. xxd 允许将二进制文件中的内容按地址顺序依次输出,可读性强
\end{itemize}
关于工具dd和xxd的使用,请见附录\ref{utilitycode}代码\ref{buildFile}
\subsection{程序流程和算法}

lipsum[1]
\subsection{程序模块及相关说明}
lipsum[1]
\section{实验过程}
测试输入数据文件和输出数据文
件
\section{实验结果} 
    \subsection{代码编译(可选)}
    在项目文件中已经包括了产生的所有代码。所以这一步可以跳过
    代码 \ref{build}展示了将汇编最终写入镜像文件的过程。
    首先使用nasm将.asm文件编译为.bin文件。随后使用dd将.bin的内容写入.img中。

    \begin{itemize}
    \item[] \begin{lstlisting}[language=sh, label=build, caption=将汇编转化为二进制并写入镜像]
    $ bximage # 先用该命令产生空白的stone.img文件
    $ sh build.sh
    \end{lstlisting}
    \end{itemize}
    代码\ref{build}的内容请见附录\ref{utilitycode}的代码\ref{buildFile}


    \subsection{运行代码}
    在这一步中,我们使用bochs运行完成好的镜像文件。bochs将自动加载相同目录内的
    .bochsrc文件,以此为配置运行虚拟机。
    \begin{itemize}
        \item[] \begin{lstlisting}[label=run, caption=运行方法]
$ sh run.sh
        \end{lstlisting}
    \end{itemize}
    
    代码\ref{run}的内容见附录\ref{utilitycode}代码\ref{runFile}

    \subsection{结果展示}
    \lipsum[2]
    
\section{实验总结}
lipsum[1]
\begin{appendices}
\section{参考文献}
\label{reference}
\begin{enumerate}
    \item The labels consists of sequential numbers.
    \item The numbers starts at 1 with every call to the enumerate environment.
  \end{enumerate}
    \section{辅助代码}
    \label{utilitycode}
    \shfilescript{buildFile}{build.sh文件内容}{../build.sh}
    \shfilescript{runFile}{run.sh文件内容}{../run.sh}
\end{appendices}
\end{document}