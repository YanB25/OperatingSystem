%%%%%%%%%%%%%%%%%%%%%%%%%%%%%%%%%%%%%%%%%
% Programming/Coding Assignment
% LaTeX Template
%
% This template has been downloaded from:
% http://www.latextemplates.com
%
% Original author:
% Ted Pavlic (http://www.tedpavlic.com)
%
% Note:
% The \lipsum[#] commands throughout this template generate dummy text
% to fill the template out. These commands should all be removed when 
% writing assignment content.
%
% This template uses a Perl script as an example snippet of code, most other
% languages are also usable. Configure them in the "CODE INCLUSION 
% CONFIGURATION" section.
%
%%%%%%%%%%%%%%%%%%%%%%%%%%%%%%%%%%%%%%%%%

%----------------------------------------------------------------------------------------
%	PACKAGES AND OTHER DOCUMENT CONFIGURATIONS
%----------------------------------------------------------------------------------------

\documentclass[a4paper]{article}

\usepackage{fancyhdr} % Required for custom headers
\usepackage{lastpage} % Required to determine the last page for the footer
\usepackage{extramarks} % Required for headers and footers
\usepackage[usenames,dvipsnames]{color} % Required for custom colors
\usepackage{graphicx} % Required to insert images
\usepackage{listings} % Required for insertion of code
\renewcommand*{\lstlistingname}{代码} % change "Listing <ref> to 代码 <ref>
\usepackage{courier} % Required for the courier font
\usepackage{lipsum} % Used for inserting dummy 'Lorem ipsum' text into the template

\usepackage[UTF8]{ctex} % Required for Chinese character
\usepackage{tocloft} % Required for beautiful toc
\usepackage[hidelinks]{hyperref} % Required for clickable toc
\hypersetup{
    colorlinks,
    citecolor=black,
    filecolor=black,
    linkcolor=black,
    urlcolor=black
}
\usepackage[title]{appendix} % Required for appendix
\usepackage{float}
\usepackage{amsmath} % used for \text{} in math formula


% used for beautiful table
\usepackage{booktabs} 
\usepackage[T1]{fontenc}
\usepackage{tabu}
\usepackage{longtable}
\usepackage[table]{xcolor}


% Margins
\topmargin=-0.45in
\evensidemargin=0in
\oddsidemargin=0in
\textwidth=6.5in
\textheight=9.0in
\headsep=0.25in

\linespread{1.1} % Line spacing

% Set up the header and footer
\pagestyle{fancy}
\lhead{\hmwkAuthorName} % Top left header
\chead{\hmwkClass\ (\hmwkClassInstructor\ \hmwkClassTime): \hmwkTitle} % Top center head
\rhead{\firstxmark} % Top right header
\lfoot{\lastxmark} % Bottom left footer
\cfoot{} % Bottom center footer
\rfoot{Page\ \thepage\ of\ \protect\pageref{LastPage}} % Bottom right footer
\renewcommand\headrulewidth{0.4pt} % Size of the header rule
\renewcommand\footrulewidth{0.4pt} % Size of the footer rule

\setlength\parindent{0pt} % Removes all indentation from paragraphs

%----------------------------------------------------------------------------------------
%	CODE INCLUSION CONFIGURATION
%----------------------------------------------------------------------------------------

\definecolor{MyDarkGreen}{rgb}{0.0,0.4,0.0} % This is the color used for comments
\lstloadlanguages{c} % Load Perl syntax for listings, for a list of other languages supported see: ftp://ftp.tex.ac.uk/tex-archive/macros/latex/contrib/listings/listings.pdf
\lstset{language=c, % Use Perl in this example
        frame=single, % Single frame around code
        basicstyle=\small\ttfamily, % Use small true type font
        keywordstyle=[1]\color{Blue}\bf, % Perl functions bold and blue
        keywordstyle=[2]\color{Purple}, % Perl function arguments purple
        keywordstyle=[3]\color{Blue}\underbar, % Custom functions underlined and blue
        identifierstyle=, % Nothing special about identifiers                                         
        commentstyle=\usefont{T1}{pcr}{m}{sl}\color{MyDarkGreen}\small, % Comments small dark green courier font
        stringstyle=\color{Purple}, % Strings are purple
        showstringspaces=false, % Don't put marks in string spaces
        tabsize=4, % 5 spaces per tab
        %
        % Put standard Perl functions not included in the default language here
        % morekeywords={rand},
        morekeywords = [1]{uint16_t, uint32_t, uint8_t, int16_t, int8_t, int32_t}
        %
        % Put Perl function parameters here
        morekeywords=[2]{on, off, interp, __attribute__},
        %
        % Put user defined functions here
        morekeywords=[3]{test},
       	%
        morecomment=[l][\color{Blue}]{...}, % Line continuation (...) like blue comment
        numbers=left, % Line numbers on left
        firstnumber=1, % Line numbers start with line 1
        numberstyle=\tiny\color{Blue}, % Line numbers are blue and small
        stepnumber=2, % Line numbers go in steps of 5,
        firstnumber=1
}

% define C style
\definecolor{main-color}{rgb}{0.6627, 0.7176, 0.7764}
\definecolor{back-color}{rgb}{0.1686, 0.1686, 0.1686}
\definecolor{string-color}{rgb}{0.3333, 0.5254, 0.345}
\definecolor{key-color}{rgb}{0.8, 0.47, 0.196}
\lstdefinestyle{mystyle}
{
    language = C++,
    basicstyle = {\small\ttfamily},
    stringstyle = {\color{Mahogany}},
    keywordstyle = {\color{blue}},
    keywordstyle = [2]{\color{Mahogany}},
    keywordstyle = [3]{\color{blue}},
    keywordstyle = [4]{\color{blue}},
    otherkeywords = {__attribute__,<<,>>,++},
    morekeywords = [2]{__attribute__},
    morekeywords = [3]{<<, >>},
    morekeywords = [4]{++, uint16_t, uint32_t, uint8_t, \#define},
}


% Creates a new command to include a perl script, the first parameter is the filename of the script (without .pl), the second parameter is the caption

\newcommand{\shfilescript}[3]{
\begin{itemize}
\item[]\lstinputlisting[caption=#2, label=lst:#1, language=sh]{#3}
\end{itemize}
}
\newcommand{\shscript}[3]{
\begin{itemize}
\item[]\begin{lstlisting}[label=lst:#1, caption=#2] #3 \end{lstlisting}
\end{itemize}
}

%----------------------------------------------------------------------------------------
%	DOCUMENT STRUCTURE COMMANDS
%	Skip this unless you know what you're doing
%----------------------------------------------------------------------------------------

% Header and footer for when a page split occurs within a problem environment
\newcommand{\enterProblemHeader}[1]{
\nobreak\extramarks{#1}{#1 见下页\ldots}\nobreak{} 
\nobreak\extramarks{接上页}{#1 见下页\ldots}\nobreak{}
}

% Header and footer for when a page split occurs between problem environments
\newcommand{\exitProblemHeader}[1]{
\nobreak\extramarks{接上页}{#1 见下页\ldots}\nobreak{}
\nobreak\extramarks{#1}{}\nobreak{}
}
% TODO:code here enable the number before section, but it disable the numbering of problems
%\setcounter{secnumdepth}{0} % Removes default section numbers
\newcounter{homeworkProblemCounter} % Creates a counter to keep track of the number of problems

\newcommand{\homeworkProblemName}{}

\newenvironment{homeworkProblem}[1][Problem \arabic{homeworkProblemCounter}]{ % Makes a new environment called homeworkProblem which takes 1 argument (custom name) but the default is "Problem #"
\stepcounter{homeworkProblemCounter} % Increase counter for number of problems
\renewcommand{\homeworkProblemName}{#1} % Assign \homeworkProblemName the name of the problem
\section{\homeworkProblemName} % Make a section in the document with the custom problem count
\enterProblemHeader{\homeworkProblemName} % Header and footer within the environment
}{
\exitProblemHeader{\homeworkProblemName} % Header and footer after the environment
}

\newcommand{\problemAnswer}[1]{ % Defines the problem answer command with the content as the only argument
\noindent\framebox[\columnwidth][c]{\begin{minipage}{0.98\columnwidth}#1\end{minipage}} % Makes the box around the problem answer and puts the content inside
}

\newcommand{\homeworkSectionName}{}
\newenvironment{homeworkSection}[1]{ % New environment for sections within homework problems, takes 1 argument - the name of the section
\renewcommand{\homeworkSectionName}{#1} % Assign \homeworkSectionName to the name of the section from the environment argument
\subsection{\homeworkSectionName} % Make a subsection with the custom name of the subsection
\enterProblemHeader{\homeworkProblemName\ [\homeworkSectionName]} % Header and footer within the environment
}{
\enterProblemHeader{\homeworkProblemName} % Header and footer after the environment
}


\newcommand{\codev}[1]{\textsf{#1}}
%----------------------------------------------------------------------------------------
%	NAME AND CLASS SECTION
%----------------------------------------------------------------------------------------

% table color
\definecolor{tableHeader}{RGB}{245, 245, 245}
\definecolor{tableLineOne}{RGB}{245, 245, 245}
\definecolor{tableLineTwo}{RGB}{224, 224, 224}
\newcommand{\tableHeaderStyle}{
    \rowfont{\leavevmode\color{white}\bfseries}
    \rowcolor{tableHeader}
}

%----------------------------------------------------------------------------------------

\newcommand{\hmwkTitle}{操作系统原理实验\ \#5} % Assignment title
\newcommand{\hmwkDueDate}{Friday,\ April\ 13,\ 2018} % Due date
\newcommand{\hmwkClass}{16级计科\ 7班} % Course/class
\newcommand{\hmwkClassTime}{周一9-10节} % Class/lecture time
\newcommand{\hmwkClassInstructor}{凌应标} % Teacher/lecturer
\newcommand{\hmwkAuthorName}{颜彬} % Your name
\newcommand{\hmwkAuthorId}{16337269} % Your id 

%----------------------------------------------------------------------------------------
%	TITLE PAGE
%----------------------------------------------------------------------------------------

\usepackage{titling}

\title{
\vspace{2in}
\textmd{\textbf{\hmwkClass:\ \hmwkTitle}}\\
\normalsize\vspace{0.1in}\small{Due\ on\ \hmwkDueDate}\\
\vspace{0.1in}\large{\textit{\hmwkClassInstructor\ \hmwkClassTime}}
\vspace{3in}
}

\author{\textbf{\LARGE{\hmwkAuthorName}} \\ \\ \textbf{\LARGE{\hmwkAuthorId}}}
\date{} % Insert date here if you want it to appear below your name
%----------------------------------------------------------------------------------------

\begin{document}
% \begin{titlingpage} % This is for ignore page number in first page. package titling

\maketitle

%----------------------------------------------------------------------------------------
%	TABLE OF CONTENTS
%----------------------------------------------------------------------------------------

% \setcounter{tocdepth}{2} % Uncomment this line if you don't want subsections listed in the ToC
% set depth in toc

% \renewcommand{\cftsecleader}{\cftdotfill{\cftdotsep}} % used for dots between <section> and <page>

\renewcommand{\contentsname}{Content} % force the word to be "content
\newpage
\tableofcontents
\addtocontents{toc}{~\hfill\textbf{Page}\par}
\newpage

% below are document body


% To have just one problem per page, simply put a \clearpage after each problem
\section{实验目的}
掌握PC微机的实模式硬件中断系统原理和中断服务程序设计方式,实现对时钟、
键盘/鼠标等硬件中断的简单服务处理程序编程和调试,让你的原型操作系统
在运行以前已有的用户程序时,能对异步事件正确地捕捉和响应。\\

掌握操作系统的系统调用原理,实现原型操作系统中的系统调用框架,提供若干
简单功能的系统调用。 \\

学习掌握C语言库的实际方法,为自己的原型操作系统配套一个C程序开发环境,
实现用自建的C语言开发简单的输入/输出的用户程序,展示封装的系统调用。
\section{实验方案}
    \subsubsection{中断调用原理}
    在实模式下,内存0x0000起始处维护着一张中断服务程序入口的表。每个
    表项站4字节,共同代表服务程序的入口地址。其中的低16位代表代码段
    地址,高16位代表段内偏移地址。\\
    
    当调用中断时,中断号代表着服务程序地址在地址表中的索引。所以当
    执行int $id$时,会将$0x0000 : id \times 4$起始的32 bits
    作为入口地址。一般寄存器ah中的值会作为中断的``功能号''。其他寄存器
    的值依据文档,相应地作为参数或作为返回值。\\
    
    当实模式下触发中断时,会首先将FLAGS寄存器压入栈中(32位模式下
    是EFLAGS),随后压入
    cs:ip,并根据地址表中的信息,跳转到相应地址。在中断返回时,要
    首先向硬件端口输出相应参数,告知中断处理硬件中断完成。随后,iret
    指令将把FLAGS和cs:ip出栈,并跳转回相应的用户程序地址。\\
    
    这些中断调用的原理就给了自定义中断的可能。用户只需修改中断地址表,
    并按照中断的行为书写自己的代码,就能实现自己的中断。 
    \subsubsection{软硬中断简介}
    软件和硬件都能引发中断。其中软件中断又称为系统调用。其与调用一个函数
    (在程序员可见角度)类似,有输入的参数和相应的返回值。程序员在系统
    调用前应手动保存需要保存的寄存器,并能预计到中断会修改相应寄存器的值。
    系统调用都是不可屏蔽的。\\ 
    
    硬件中断有可屏蔽和不可屏蔽两种。其中时钟中断(watchdog)是
    不可屏蔽中断。由于硬件中断可以发生在任何一个时间点,所以硬件中断
    应能保证绝对不修改任何寄存器的值。即硬件中断对程序员是透明的。
    \subsubsection{用户自定义中断}
    用户自定义中断可以采用如下的方式。首先查阅中断地址表,找到未被
    占用的中断号(例如本项目使用的0x2B号)。将地址表的相应表项修改
    用户自定义的地址。则当中断发生时,程序控制流可以到达用户自定义地址。\\
    
    标准的中断实现中,用户自定义中断应该提取ah作为功能号,建立中断自己的
    功能地址表,根据ah的值在功能地址表中提取相应功能的中断入口地址,并跳转
    到该地址。\\
    
    中断返回时,要保证将栈退到``刚进入中断''时的状态,向硬件端口传送一些
    数据,并用iret指令返回。
    \subsection{实验环境}
    \subsubsection{系统与虚拟机}
    \begin{itemize} \item 操作系统 \\ 
        本实验在Linux下完成。采用Ubuntu 16.04
        \item 虚拟机\\
        bochs.它是一款开源且跨平台的 IA-32 模拟器。
    \end{itemize}
    \subsubsection{相关工具、指令}
        \begin{itemize}
            \item 汇编器\\ 
            NASM. NASM是一个轻量级的、模块化的 80x86 和 x86-64 汇编器。它的语法与
            Intel 原语法十分相似,但更加简洁和易读。它对宏有十分强大的支持。
            \item 编译器\\
            GCC. GNU/GCC 是开源的C语言编译器。其产生的伪16位代码可以与NASM结合,混编生成伪16位程序。
            \item 镜像文件产生工具\\ 
            bximage. 该命令允许生成指定大小的软件镜像。
            \item 二进制写入命令\\ 
            dd. dd 允许指定源文件和目标文件,将源文件的二进制比特写入目标文件中的指定位置。
            \item 二进制文件查看命令\\ 
            xxd. xxd 允许将二进制文件中的内容按地址顺序依次输出,可读性强
            \item 反汇编器\\
            objdump. objump可以查看目标文件和二进制文件的反汇编代码,还能指令intel或at\&t格式显示。
            \item 代码生成脚本\\
            makefile. makefile脚本具有强大的功能,其可以识别文件依赖关系,自动构筑文件,自动执行shell脚本等。
        \end{itemize}
 
\section{实验过程}
    \subsubsection{GCC语言拓展的运用}
    为了简化代码的书写,
    \subsection{修改时钟中断}
    \subsection{修改键盘中断}
    \subsection{安装用户自定义中断}
    \begin{table}[!htb]
        \caption{GCC编译指令}\label{tab:gcc}
    \begin{tabular}{@{} *5l @{}} 
        \toprule
    \emph{编译选项} & \emph{作用} &&&  \\
        \midrule
        -o    & D  \\ 
        \\ 
        -march=i386  & X产生原始Itel i376 CPU架构的代码\\ 
        \\
        -m16& 等价于汇编伪指令.code16gcc。
            \\& 在所有必要的指令前加,前缀66, 67,
            \\& 使得处理器能以32位的方式读取GCC产生的代码\\
        \\
        -mpreferred-stack-boundary=2 & 以$2^i$bytes作为对齐量,对齐栈的界限。
            \\& 该选项默认为4(即默认$2^4$字节对齐栈界限).
            \\& 为了节省空间,这里设置成2 \\
        \\
        -ffreestanding &  产生``自立''的程序文件.
            \\& 即告知GCC不使用(几乎)所有的库头文件。
            \\& 保证产生的代码不依赖于GCC的自带库 \\
        \\
        -c & 不产生可执行文件,只执行``编译''操作产生目标文件 \\ 
        \\
        -Og & 产生最小限度的优化.
            \\& 即这个优化能化简汇编代码,但又
            不影响汇编程序的可读性。\\
        \\
        -O0 & 不产生任何优化。
            \\& 本项目文件系统FAT表的构建使用C语言的struct
            完成。
            \\& 为了保证产生的二进制表项和C代码中的struct严格一致
            \\& 必须
            采用本优化等级。\\

        \bottomrule
    \hline
    \end{tabular}
    \end{table}

\section{实验总结}
    \subsection{亮点介绍}
   

    \subsection{心得体会}
 
    \subsection{BUG汇总}
\begin{appendices}
\section{参考文献} \label{sec:reference}
sasadf
\begin{enumerate}
    \item https://blog.csdn.net/longintchar/article/details/50602851 \\
    16位和32位汇编指令的不同(尤其是push指令)
    \item https://www.ibiblio.org/gferg/ldp/GCC-Inline-Assembly-HOWTO.html\#s1 \\
    GCC 内嵌汇编的书写。
    \item http://blog.51cto.com/dengqi/1349327
    FAT32文件系统讲解
    \item https://blog.csdn.net/yeruby/article/details/41978199
    FAT16文件系统讲解
  \end{enumerate}
\section{其他代码} \label{sec:otherCode}
\end{appendices}
\end{document}